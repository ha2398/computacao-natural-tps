\documentclass[12pt]{article}

\usepackage{sbc-template}

\usepackage{graphicx,url}

\usepackage[brazil]{babel}   
%\usepackage[latin1]{inputenc}  
\usepackage[utf8]{inputenc}  
% UTF-8 encoding is recommended by ShareLaTex

\usepackage{amssymb}
     
\sloppy

\title{Trabalho Prático 2 - Algoritmo de Colônia de Formigas para o problema da p-Mediana}

\author{Hugo Araujo de Sousa}

\address{
  Computação Natural (2017/2) \\
  Departamento de Ciência da Computação \\
  Universidade Federal de Minas Gerais (UFMG)
  \email{hugosousa@dcc.ufmg.br}
}

\begin{document} 

\maketitle
     
\begin{resumo}
  Este trabalho objetiva o desenvolvimento de conceitos fundamentais na
  aplicação do Algoritmo de Colônia de Formigas (Ant Colony Optimization - ACO)
  para resolução do problema da p-Mediana. A estrutura básica do ACO é apresentada
  e adaptada ao problema a ser resolvido. Finalmente, a partir de dados de teste,
  são realizados experimentos e a análise dos resultados obtidos.
\end{resumo}

\section{INTRODUÇÃO}

Como definido por \cite{clalg:11}, o Algoritmo de Colônia de Formigas é um método
dos campos de Inteligência de Enxames, Metaheurísticas e Inteligência Computacional.
Nesse método, o comportamento de formigas na natureza, em particular a comunicação
baseada em feromônio que elas realizam para encontrar bons caminhos na busca por comida
em um ambiente, é inspiração para encontrar potenciais soluções para um problema.

Na busca por comida, formigas se espalham inicialmente aleatoriamente em seu ambiente.
Uma vez que uma fonte de comida é localizada, as formigas que a encontraram começam a
depositar feromônio nesse ambiente, marcando assim o caminho que as levaram até a fonte.
Quando várias um mesmo caminho é percorrido várias vezes e por várias formigas, a quantidade
de feromônio nesse caminho se torna notavelmente maior do que em outras partes do ambiente,
enquanto caminhos pouco percorridos perdem feromônio à medida que o tempo passa, devido à
evaporação do mesmo.A comunicação das formigas, e ponto fundamental para o algoritmo,
ocorre através do feromônio depositado, uma vez que elas tendem a seguir por caminhos com
maior quantidade de feromônio.

Usando esse comportamento como inspiração, surge o Algoritmo de Colônia de Formigas, cuja 
estratégia geral é a de construir soluções candidatas para um problema de forma estocástica.
Essas soluções são construídas, então, de forma probabilística e têm suas qualidades avaliadas.
A partir dessas medidas de qualidade, 'feromônio' é depositado nos caminhos que geraram soluções
de maior qualidade e, dessa forma, novas soluções criadas tendem a seguir pelos mesmos caminhos.

Nesse trabalho, o Algoritmo de Colônia de Formigas será utilizado para resolver o problema da p-Mediana
com restrições de capacidade. Esse problema consiste em decidir onde localizar \textit{p} centros em uma
rede (composta por vértices e arestas) de forma a minimizar a soma de todas as distãncias de cada vértice
ao centro mais próximo. Nesse problema também existem restrições de capacidade de atendimento dos centros.
Esse problema é um problema de otimização combinatória NP-difícil.

\section{MODELAGEM} \label{sec:model}

Para utilizar o Algoritmo de Colônia de Formiga para o problema da p-Mediana, é necessário realizar uma
modelagem relativa à construção de soluções, comportamento de formigas, depósito de feromônio, entre outros
aspectos. No trabalho em questão, a modelagem seguiu em grande parte \cite{DBLP:journals/informaticaSI/FrancaZC05}.

\subsection{Indivíduos} \label{sec:ind}

A primeira decisão de implementação em Programação Genética é a de como representar
os indivíduos que representarão soluções para o problema a ser resolvido. Para o
problema de Regressão Simbólica, uma solução é uma função do tipo
$ f:\mathbb{R}^{m} \rightarrow \mathbb{R} $. Dessa forma, a representação escolhida
para representar uma função que resolva o problema é a de uma árvore onde os nós
internos representam operadores e os nós folha são formados por terminais, isto é,
variáveis da função ou constantes. Para os nós que representam operadores, os nós filhos
serão os operandos. A Figura \ref{fig:gp_ind} mostra um exemplo de indivíduo com essa
configuração.

Para este trabalho, o conjunto de terminais e operadores escolhidos é mostrado a seguir:

\begin{itemize}
 \item \textbf{Operadores:}
 \begin{itemize}
  \item \textbf{Binários:}
   \begin{itemize}
    \item \textbf{+}: Operador de adição.
    \item \textbf{-}: Operador de subtração.
    \item \textbf{*}: Operador de multiplicação.
    \item \textbf{/}: Operador de divisão.
    \item \textbf{\^}: Operador de exponenciação.
   \end{itemize}
   
   \item \textbf{Unários:}
   \begin{itemize}
    \item \textbf{log}: Função logaritmo (base e).
    \item \textbf{sin}: Função seno.
    \item \textbf{cos}: Função cosseno.
    \item \textbf{sqrt}: Função raiz quadrada.
   \end{itemize}

 \end{itemize}
 
 \item \textbf{Terminais:}
 \begin{itemize}
  \item \textbf{Constantes}: Valores reais aleatoriamente gerados.
  \item \textbf{Variáveis}: Representam variáveis da função a ser aproximada.
  O número de variáveis aleatoriamente geradas é igual ao número de variáveis
  de entrada da função que o usuário deseja aproximar.
 \end{itemize}
 
\end{itemize}

\subsection{Fitness}

Como dito anteriormente, cada indivíduo presente em uma determinada geração
deverá ser avaliado para obter uma medida de quão bem esse indivíduo aproxima
a função cujos valores de entrada e saída são fornecidos pelo usuário. O critério
de avaliação da qualidade do indivíduo, também conhecido como fitness, para este
trabalho, será dado pela raiz quadrada do erro quadrático médio (RMSE).

\begin{center}
 \begin{math}
  f(Ind) = \sqrt{\frac{1}{N}\sum_{n=1}^{N}(Eval(Ind, x) - y)^2}
  \end{math}
\end{center}

onde Ind é o indivíduo sendo avaliado, $ Eval(Ind, x) $ avalia o indivíduo Ind
no conjunto de entrada fornecido $ x $, $ y $ é a saída correta da função para 
a entrada $ x $ e $ N $ é o número de exemplos fornecidos.

Dessa forma, a função $ Eval(Ind, x) $ vai atribuir os valores em $ x $ às
variáveis presentes em Ind, e retornar o valor $ y $ correspondente à saida
da expressão simbólica que Ind representa para $ x $.

\subsection{Geração de População Inicial}

Existem vários métodos para geração da população inicial de indivíduos que serão
evoluídos ao longo das gerações em que o programa executará. Neste trabalho é usada
uma combinação de dois deles:

\begin{itemize}
 \item \textbf{Full:} Gera indivíduos cuja expressão simbólica é dada por uma árvore
 completa, i.e., todas as folhas estão na mesma profundidade.
 \item \textbf{Grow:} Gera indivíduos com formas e tamanhos variados, uma vez que
 os nós são selecionados tanto dos operadores quanto dos terminais até que a profundidade
 limite é atingida. Quando isso acontece, somente terminais são selecionados para compor
 nós.
\end{itemize}

Para garantir uma diversidade elevada de indivíduos na população inicial, é usado o
método \textbf{Ramped Half-and-Half}, que combina os métodos Full e Grow, gerando
metade da população com o primeiro e metade com o segundo.

\subsection{Evolução}

Uma vez gerada a população de indivíduos, o algoritmo de Programação Genética entra
em um laço de repetição, evoluindo indivíduos e gerando populações cada vez melhores.
A evolução dos indivíduos presentes em uma população se dá através de seleção e aplicação
de operadores genéticos.

\subsubsection{Seleção}

É desejável que, a cada geração, sejam mantidas as características dos indivíduos que
apresentam melhor fitness (menor, no caso da regressão simbólica). Logo, é necessário
incluir um mecanismo que permita selecionar, dada uma certa população, indivíduos que
se destacam.

Nesse trabalho, o mecanismo de seleção implementado foi o de \textbf{Seleção por Torneio}.
Esse tipo de seleção escolhe um grupo aleatório de tamanho $ k $ dentre os indivíduos
de uma população e retorna aquele, dentre os $ k $ escolhidos, que tenha a melhor fitness.

Além disso, foi adicionado ao projeto a possibilidade de se utilizar o conceito de 
\textbf{elitismo}. Com esse conceito, os $ n $ melhores indivíduos de cada geração são
simplesmente passados para a geração seguinte, onde $ n $ é um parâmetro definido pelo
usuário.

\subsubsection{Operadores Genéticos}

A partir dos indivíduos selecionados em cada geração, aplica-se um conjunto de operadores
genéticos sobre os mesmos, a fim de garantir que novos indivíduos (baseados nesses que
se destacaram em cada geração) possam surgir. Cada um desses operadores genéticos contribui
de uma forma para o algoritmo e é aplicado seguindo uma certa probabilidade pré-definida.

Os operadores genéticos implementados nesse trabalho são apresentados a seguir.

\begin{itemize}
 \item \textbf{Cruzamento:} Dois indivíduos são selecionados e partes desses indivíduos
 (subárvores) são trocadas de lugar, gerando dois novos indivíduos. Os indivíduos pais
 não são alterados nesse processo.
 
 \item \textbf{Mutação:} Uma parte aleatória de um indivíduo selecionado é alterada.
 Essa parte pode ser removida, expandida, simplesmente ter seu conteúdo trocado, etc.
 Para o trabaho, foi implementada mutação de subárvore, onde um nó é escolhido para
 ser expandido, sendo criada uma nova subárvore a partir do mesmo.
 
 \item \textbf{Reprodução:} Um indivíduo selecionado é simplesmente passado adiante para
 a próxima geração.
\end{itemize}


\section{IMPLEMENTAÇÃO}

Dada a modelagem do problema mostrada na Seção \ref{sec:model} o algoritmo principal
segue o pseudocódigo mostrado na Figura \ref{fig:gp_code}.

Para o trabalho, o algoritmo então foi implementado usando a linguagem Python 3.

A classe \textbf{Individual} representa um indivíduo conforme descrito na Seção
\ref{sec:ind}, e é a principal do projeto. Nela estão presentes todos os métodos 
para construção de indivíduos, populações, avaliação de indivíduos, etc.

Algumas decisões de implementação importantes são discutidas a seguir:

\begin{itemize}
 \item Biblioteca Numpy utilizada no projeto intensivamente. É com ela que toda a geração
 de números aleatórios é feita e as operações dos nós são feitas.
 
 \item As constantes geradas para nós terminais são números reais no intervalo
 $ [-10, 10] $.
 
 \item A fim de otimizar a cópia dos indivíduos pais durante a etapa de cruzamento,
 foi criada uma função copy, mais específica do que a alternativa deepcopy da linguagem.
 
 \item No método de criação de indivíduos Grow, os tipos dos nós são gerados aleatoriamente
 com probabilidades $ 0.8, 0.1 $ e $ 0.1 $ para nós de operadores, variáveis e constantes,
 respectivamente.
 
 \item Para evitar que os indivíduos gerados cresçam indefinidamente, existe uma penalidade
 para indivíduos que excedam um tamanho limite. Esse tamanho (número de nós) máximo é dado
 por $ 2^(D) - 1 $, onde $ D $ é a profundidade máxima das árvores geradas. Indivíduos que
 ultrapassem esse tamanho recebem fitness infinita.
 
 \item A produndidade máxima dos indivíduos não é variada, tendo valor $ 7 $.
\end{itemize}


\section{ESTRUTURA DO PROJETO E EXECUÇÃO}

Os arquivos de código-fonte do projeto se encontram na pasta \textbf{src}. Nela, o
código-fonte é dividido da seguinte forma:

\begin{itemize}
 \item \textbf{individual.py:} Define a classe Individual e todos os métodos necessários
 para criação e avaliação de indivíduos e criação de populações.
 
 \item \textbf{gp.py:} Métodos auxiliares para Programação Genética, tais como seleção
 por torneio, avaliação de população e operadores genéticos.
 
 \item \textbf{tp1.py:} Programa principal. Nele está implementado o algoritmo principal
 de Programação Genética, além de todos os métodos de manipulação de entrada e saída.
\end{itemize}

\subsection{Execução e Parâmetros}

A fim de facilitar a execução do programa, foram definidos parâmetros para alterar 
alguns pontos chave para Programação Dinâmica. São eles:

\begin{itemize}
 \item \textbf{Semente:} Número inteiro usado na geração de números aleatórios durante a 
 execução do programa. Note que, mantendo todos os outros parâmetros fixos, a saída
 do programa para uma mesma entrada e valor de seed será sempre igual. A semente
 da execução pode ser definida com a flag \textbf{-s} e tem valor padrão $ 0 $.
 
 \item \textbf{Tamanho da população:} Número inteiro que determina o tamanho de indivíduos
 presente em cada geração. Pode ser definido com a flag \textbf{-p} e tem valor padrão
 $ 54 $.
 
 \item \textbf{Tamanho do torneio:} Número de indivíduos que competem em cada seleção
 por torneio. Definido com a flag \textbf{-k} e tem valor padrão $ 7 $.
 
 \item \textbf{Número de gerações:} Define o número de gerações pelo qual o programa
 deve executar. Definido com a flag \textbf{-g} e tem valor padrão $ 10 $.
 
 \item \textbf{Taxa de cruzamento:} Probabilidade de usar o operador de cruzamento para
 gerar filhos em cada geração. Definida com a flag \textbf{-c}, valor padrão $ 0.9 $.
 
 \item \textbf{Taxa de mutação:} Probabilidade de usar o operador de mutação para
 gerar filhos em cada geração. Definida com a flag \textbf{-m}, valor padrão $ 0.05 $.
 
 \item \textbf{Tamanho da elite:} Tamanho da elite a ser transferida automaticamente para
 a próxima geração, a cada geração. Definido com a flag \textbf{-e}, valor padrão $ 2 $.
 Note que para desativar o elitismo, basta usar \textbf{-e 0}.
 
\end{itemize}

Note que a taxa de reprodução não é definida pelo usuário, mas calculada em função das 
taxas de cruzamento e mutação, como mostrado na Figura \ref{fig:gp_code}.

Para executar o programa, o comando abaixo é usado:

\begin{center}
 tp1.py [\-h] [\-s RSEED] [\-p POP\_SIZE] [\-k KTOUR] [\-g NGEN] [\-c CROSSR]
              [\-m MUTR] [\-e ELIT]
              train\_file test\_file
\end{center}

Onde RSEED indica a semente, POP\_SIZE o tamanho da população, KTOUR o número de competidores
em torneios, NGEN o número de gerações, CROSSR a probabilidade de cruzamento, MUTR a probabilidade
de mutação, ELIT o tamanho da elite em cada geração, train\_file o nome do arquivo com as
entradas de treino e test\_file o nome do arquivo com as entradas de teste.

Todos os parâmetros entre colchetes acima são opcionais.

\subsection{Entrada e Saída}

Os arquivos de entrada, tanto de treino quanto de teste, devem possuir a mesma estrutura.
O arquivo de treino é usado para evoluir as soluções do programa até o número máximo de
gerações ser alcançado. Quando essa etapa é finalizada, as melhores soluções encontradas
são avaliadas com o arquivo de teste.

Em ambos os arquivos, cada linha representa uma amostra da função a se aproximar. Sendo
$ x + 1 $ valores de ponto flutuante separados por vírgula, onde $ x $ é o número de variáveis
da função. A última coluna, então, representa a saída $ y $ da função para os valores de entrada
das colunas anteriores.

Em relação à saída do programa, é impresso, a cada geração, um resumo de estatísticas.
Primeiramente é impresso o número da geração atual, em seguida o valor da fitness do melhor
indivíduo da geração, da fitness do pior indivíduo, a fitness média considerando todos
os indivíduos da geração, o número de indivíduos repetidos, a taxa de melhoria para mutação
e cruzamento.

As taxas de melhoria são calculadas da seguinte forma:

\begin{center}
 \begin{math}
  Imp(op) = \frac{NImp}{NGen}
 \end{math}
\end{center}

Sendo Imp(op) a taxa de melhoria para o operador op, NGen representa o número de indivíduos
criados em determinada geração utilizando-se o operador genético op e NImp o número, dentre
esses indivíduos gerados, daqueles que apresentaram fitness melhor do que seus pais (indivíduos
sobre os quais o operador op foi aplicado).

Exemplo de estatísticas para uma certa geração:

\begin{center}
  Generation 7\\
  Best fitness: 2.77477679664e-09\\
  Worst fitness: inf\\
  Average fitness: 1.70803185868e+17\\
  Number of repeated individuals: 15\\
  Mutation improvement rate: 0.0\% \\
  Crossover improvement rate: 6.25\%
\end{center}

\section{EXPERIMENTOS}

Nessa seção serão apresentados os experimentos realizados. Todos eles foram executados em
uma máquina Intel Core i7-5500U, 2.40GHz, 4 cores, 8GB de memória RAM e sistema operacional
ubuntu 14.04 LTS.

\subsection{Metodologia}

Muitas instâncias de teste foram executadas para cada uma das bases de teste. Alguns exemplos
de saídas obtidas estão presentes na pasta \textbf{test}.

Um script para teste de todas as bases presentes no diretório \textbf{datasets} com todas as
possíveis configurações de parâmetro e 30 execuções/sementes foi desenvolvido. Para executá-lo,
basta digitar o comando a seguir no terminal:

\begin{center}
 ./run\_tests.py
\end{center}

A metodologia para execução dos testes seguiu muitas das orientações presentes em \cite{clalg:11}.

\subsection{Experimentos}

Abaixo são mostrados os principais experimentos realizados. Os resultados são apresentados
na Seção \ref{sec:result}. Os valores apresentados foram obtidos da média de 30 execuções.

\begin{itemize}
 \item \textbf{Experimento 1:} No primeiro experimento, objetivou-se simplesmente a verificação
 de convergência da melhor indivíduo em relação ao número de gerações.Os parâmetros utilizados
 estão listados abaixo. \\
 
 \begin{itemize}
  \item Base keijzer-7:
  
  Tamanho da população: \\
  Competidores em torneios: \\
  Número de gerações: \\
  Taxa de cruzamento: \\
  Taxa de mutação: \\
  Tamanho da elite: \\
 
  \item Base keijzer-10:
  
  Tamanho da população: 100\\
  Competidores em torneios: 2\\
  Número de gerações: 100\\
  Taxa de cruzamento: 0.8\\
  Taxa de mutação: 0.1\\
  Tamanho da elite: 2\\
  
  \item Base house:
  
  Tamanho da população: 100\\
  Competidores em torneios: 7\\
  Número de gerações: 50\\
  Taxa de cruzamento: 0.9\\
  Taxa de mutação: 0.05\\
  Tamanho da elite: 2\\
  
 \end{itemize}
 
 \item \textbf{Experimento 2:} Com o experimento 2, procurava-se determinar a melhor configuração,
 dos parâmetros de número de gerações e tamanho de população inicial. Para isso, a base house foi
 utilizada. Os parâmetros são mostrados abaixo.
 
 Base house \\
 Tamanho da População: \{5, 100, 500\} \\
 Competidores em torneios: 7 \\
 Número de gerações: 50 \\
 Taxa de cruzamento: 0.8 \\
 Taxa de mutação: 0.1 \\
 Tamanho da elite: 2 \\
 
 \item \textbf{Experimento 3:} Já com valores de tamanho de população e número de gerações fixados,
 o próximo passo foi determinar os parâmetros de probabilidade de aplicação dos operadores genéticos.
 
 Foram testadas duas configurações:
 
 Base house \\
 Tamanho da População: 50 \\
 Competidores em torneios: 7 \\
 Número de gerações: 50 \\
 Cruzamento, mutação: \{0.9, 0.05\} e \{0.6, 0.3\} \\
 Tamanho da elite: 2 \\
 
 \item \textbf{Experimento 4:} Uma vez estabelecidas as taxas de cruzamento e mutação que promovem
 um melhor resultado do programa, é necessário testar os valores de tamanho de torneio. Esse valor
 refere-se ao número de competidores que competem a cada seleção de indivíduos por torneio.
 
 Base keijzer-7 \\
 Tamanho da População: 50 \\
 Competidores em torneios: 3 e 7 \\
 Número de gerações: 50 \\
 Taxa de cruzamento: 0.9 \\ 
 Taxa de mutação: 0.05 \\
 Tamanho da elite: 2 \\
 
 \item \textbf{Experimento 5:} Em seguida, vemos como é o comportamento do melhor indivíduo de
 cada geração, e, consequentemente, da melhor fitness, utilizando ou não de elitismo. Parâmetros
 utilizados:
 
 Base house \\
 Tamanho da População: 50 \\
 Competidores em torneios: 7 \\
 Número de gerações: 50 \\
 Taxa de cruzamento: 0.9 \\ 
 Taxa de mutação: 0.05 \\
 Tamanho da elite: 0 e 2 \\
 
 \item \textbf{Experimento 6:} Uma vez que a diversidade dos indivíduos presentes em uma população
 cai muito, a melhor fitness de cada geração tende a permanecer sem grandes modificações. Para este
 trabalho, a única medida, ainda que não muito eficaz, de diversidade, é o número de indivíduos
 repetidos. Um indivíduo é considerado repetido aqui quando já existe na mesma geração um indivíduo
 que apresenta a mesma estrutura (mesma expressão simbólica - genótipo do indivíduo).
 
 Base keijzer-7 \\
 Tamanho da População: 500 \\
 Competidores em torneios: 7 \\
 Número de gerações: 50 \\
 Taxa de cruzamento: 0.9 \\ 
 Taxa de mutação: 0.05 \\
 Tamanho da elite: 2 \\
 
\end{itemize}


\section{RESULTADOS} \label{sec:result}

\subsection{Experimento 1: Convergência de melhor fitness}

Esse experimento comprova que, de fato, o algoritmo implementado leva as populações
geradas inicialmente a convergirem eventualmente, tentando sempre se aproximar do valor
ótimo de fitness igual a zero para as bases keijzer (o que indica que a solução ótima
pode ter sido encontrada).
  
Podemos ver que, em um primeiro momento a melhor fitness cai de forma bem acentuada,
em seguida convergindo a um valor específico para cada uma das bases.

\subsection{Experimento 2: Tamanho de população e número de gerações}

A partir do experimento 2 foi possível constatar que o tamanho da população tem grande impacto
na melhor fitness das primeiras gerações. Isso se deve ao fato de que aumentando o número de
invivíduos estamos aumentando a diversidade e explorando melhor o espaço de busca. Também
podemos observar na Figura \ref{fig:exp2h} que, para todos os valores de tamanho de população,
o programa converge para valores muito próximos, e, além disso, isso ocorre sempre por volta da 
geração 30. Concluímos então que o custo adicional de avaliar mais indivíduos não compensa muito
quando vemos que o ponto de convergência é muito próximo para os valores de tamanho de população
testados.

\subsection{Experimento 3: Operadores Genéticos}

Com esse experimento podemos observar como a velocidade de convergência é afetada diretamente pela
escolha dos parâmetros relativos às probabilidades de uso de cada um dos operadores genéticos.

Vemos que, ao utilizar uma configuração onde a taxa de cruzamento é muito elevada (garantindo
que gerações posteriores possam apresentar grandes semelhanças com as os indivíduos de melhor
fitness das gerações anteriores) e a de mutação é reduzida (somente para garantir que haja uma certa
diversidade para explorar o espaço de busca) o programa não só converge mais rapidamente, como
chega a um valor bem melhor de fitness. A Figura \ref{fig:exp3h} mostra os resultados para esse
experimento.

\subsection{Experimento 4: Torneio}

No experimento 4, ilustrado na Figura \ref{fig:exp4k7}, vemos que é importante selecionar um tamanho de
torneio que seja grande o suficiente para incluir bons indivíduos, porém não muito grande que sempre 
selecione aqueles de maior fitness (condição importante para manter diversidade).


\subsection{Experimento 5: Elitismo}

Com esse experimento vemos que o uso do elitismo é importante para garantir que a solução seja sempre
melhorada ou mantenha-se constante. Sem o uso do mesmo, a melhor fitness pode piorar de uma geração
para a próxima. A Figura \ref{fig:exp5h} ilustra o experimento.

\subsection{Experimento 6: Indivíduos repetidos}

O resultado desse experimento, como indicado na Figura \ref{fig:exp6k7}, mostra que, uma vez que o
programa encontra uma solução muito boa (próxima da solução ótima), todos os outros indivíduos de
uma geração tendem ter seus genótipos aproximados a essa solução.

\subsection{Observações}

Os experimentos realizadas possibilitaram observar diversas características importantes, tanto
de Programação Genética em geral, quanto da implementação desse trabalho e das bases de dados
utilizadas.

Vemos que a escolha da probabilidade dos operadores genéticos e tamanho de torneio pode melhorar
a diversidade dos indivíduos, promovendo uma maior exploração do espaço de busca.

Além disso, à medida que a melhor fitness de uma população converge para um determinado valor,
é de se esperar que o número de indivíduos repetidos também tenda a aumentar, dada uma taxa 
de cruzamento alta.

Um ponto importante sobre os dados coletados diz respeito à discrepância entre os valores de 
melhor, pior e fitness média das gerações. Mesmo que uma determinada geração possua melhor fitness
muito baixa, nada impede que a fitness média e pior sejam muito elevadas. De fato, é isso que se
espera quando utilizamos operadores como multiplicação e exponenciação no conjunto de operadores
dos indivíduos.

\section{CONCLUSÃO}

O trabalho aqui apresentado foi de grande utilidade para fixar vários conceitos vistos em aula
na disciplina de Computação Natural. Creio que a habilidade de modelar um problema como uma
instância de Programação Genética foi reforçada e poderá ser aplicada com maior facilidade
no futuro.

Pontos chave de aprendizado devem ser ressaltados. Na Programação Genética é importante
definir como representar os indivíduos que representam soluções, e, para isso, é importante
identificar como mapear soluções já conhecidas para soluções genéricas.

Além disso, pode ser constatado que a escolha dos parâmetros do algoritmo afetam significativamente
os resultados obtidos. Sendo o processo de teste algo iterativo, é importante definir a cada
momento a configuração de parâmetros que afeta de maneira mais positiva a saída do programa.

A maior dificuldade encontrada foi lidar com a grande quantidade de testes necessária para
avaliar o projeto. Visto que cada geração pode demorar muito tempo para executar (cada indivíduo
deve ser avaliado e o tamanho da entrada afeta a performance diretamente), foi necessário começar
o processo de testes assim que possível para obter dados e poder analisá-los.

De maneira geral, o aprendizado obtido tem grande apelo, não só para conceitos de Programação Genética,
mas também de Computação Evolucionária em geral.

\section{REFERÊNCIAS}

\bibliographystyle{sbc}
\bibliography{sbc-template}

\end{document}
