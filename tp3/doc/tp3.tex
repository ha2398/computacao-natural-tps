\documentclass[12pt]{article}

\usepackage{sbc-template}

\usepackage{graphicx,url}

\usepackage[brazil]{babel}   
%\usepackage[latin1]{inputenc}  
\usepackage[utf8]{inputenc}  
% UTF-8 encoding is recommended by ShareLaTex

\usepackage{mdframed}
\usepackage{minted}
\usepackage{hyperref}
\usepackage{amssymb}
     
\sloppy

\title{Trabalho Prático 3 - Redes Neurais Artificiais}

\author{Hugo Araujo de Sousa}

\address{
  Computação Natural (2017/2) \\
  Departamento de Ciência da Computação \\
  Universidade Federal de Minas Gerais (UFMG)
  \email{hugosousa@dcc.ufmg.br}
}

\begin{document} 

\maketitle
     
\begin{resumo}
  Nesse trabalho são explorados conceitos relacionados a redes neurais,
  colocando-os em prática através da utilização da biblioteca Keras com Tensorflow, 
  que nos permite abordar um problema de classificação.
\end{resumo}

\section{INTRODUÇÃO}

Dentro da área de Computação Natural, o campo de Redes Neurais Artificiais
tem como objetivo a criação de modelos computacionais inspirados pelo
conhecimento que temos sobre como funciona o sistema nervoso, mais
especificamente, na estrutura e função dos neurônios no cérebro
\cite{clevalg}.

No trabalho em questão, usaremos a biblioteca Keras 
\footnote{\url{https://keras.io/}} com Tensorflow
\footnote{\url{https://www.tensorflow.org/}}, que juntas fornecem
implementações de redes neurais já prontas para uso. A partir dessas 
duas bibliotecas, o problema a ser resolvido será o de classificação
de um conjunto de dados específico.

Esse conjunto de dados reúne informações de 1429 proteínas, descritas
por 8 atributos (números reais). Para cada uma dessas proteínas, a sua
classe se refere à parte da célula em que a proteína se encontra.
Ao todo, existem 7 classes possíveis, descritas na Tabela
\ref{tab:classes}.

\begin{table}[h]
	\centering
	\begin{tabular}{|c|c|}
		\hline
		\textbf{Classe} & \textbf{Descrição} \\ \hline
		\textbf{CYT} & Citoplasma \\ \hline
		\textbf{MIT} & Mitocôndria \\ \hline
		\textbf{ME1} & Uma membrana específica da célula \\ \hline
		\textbf{ME2} & Uma membrana específica da célula \\ \hline
		\textbf{ME3} & Uma membrana específica da célula \\ \hline
		\textbf{EXC} & Exterior da célula \\ \hline
		\textbf{NUC} & Núcleo da célula \\ \hline
	\end{tabular}
	\caption{\label{tab:classes} Classes que descrevem a posição
	das proteínas em uma célula.}
\end{table}

\section{MODELAGEM}



\section{IMPLEMENTAÇÃO}



\section{ESTRUTURA DO PROJETO E EXECUÇÃO}



\section{EXPERIMENTOS}



\section{CONCLUSÃO}



\section{REFERÊNCIAS}



\bibliographystyle{sbc}
\bibliography{sbc-template}

\end{document}
